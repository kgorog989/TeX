\documentclass[aspectratio=169, bigger, xcolor={table}]{beamer}
\usepackage[magyar]{babel}
\usepackage{t1enc}
\usepackage{lipsum}
\usepackage{hulipsum}
\usepackage{mathtools}
\usepackage{colortbl}
\usepackage{amsthm}
\usepackage{enumerate}

\usetheme{CambridgeUS}
\usecolortheme{spruce}

\author{Görög Krisztina Erzsébet}
\title{Feladatok megoldása}
\subtitle{A feladat megoldás}
\institute{Miskolci Egyetem}
\date{\today}



\begin{document}
\maketitle

\begin{frame}{A frame címe}{Ez az alcím}
Ez itt ennek a frame-nek a tartalma.

A feladat szerint írni kell néhány sort.
\end{frame}

\begin{frame}[allowframebreaks]{Második frame}{2. alcím}
\hulipsum
\end{frame}

\begin{frame}{Harmadik frame}{Harmadik frame alcíme}
\begin{columns}[c]

\begin{column}{0.5\linewidth}
\begin{enumerate}
\item klsda
\item lksd
\item klasd
\item lksdf
\end{enumerate}

\begin{itemize}
\item kds
\item klsdjf
\item ksldkf
\end{itemize}
\end{column}

\begin{column}{0.5\linewidth}

\end{column}
fdgh
\end{columns}
\end{frame}

\begin{frame}{Negyedik frame}{Negyedik frame alcíme}
Ez itt ennek a frame-nek a tartalma.

A feladat szerint írni kell néhány sort.
\end{frame}

\begin{frame}[fragile]{Verbatimos frame}{Verbatim alcím}

\begin{verbatim}
\fjkd
slskjf
\ksdjfk
\ksdj

\end{verbatim}

\end{frame}

\end{document}