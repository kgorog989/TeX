\documentclass{article}
\usepackage[magyar]{babel}
\usepackage{t1enc}
\usepackage{lipsum}
\usepackage{hulipsum}
\usepackage{amsthm}

\newtheorem{tet}{Tétel}
\theoremstyle{definition}
\newtheorem{defin}{Definíció}
\theoremstyle{plain}
\newtheorem{lemma}[tet]{Lemma}
\theoremstyle{remark}
\newtheorem{feladat}{Feladat}[section]
\newtheorem*{megj}{Megjegyzés}

\begin{document}

\section{Első section}
\begin{defin}[Első]
A derékszögű háromszög...
\end{defin}
\begin{lemma}
Ha egy háromszögnek...
\end{lemma}

\begin{feladat}[1]
Legyenek a háromszög oldalai...
\end{feladat}
\begin{feladat}[2]
Legyenek a háromszög oldalai...
\end{feladat}
\begin{feladat}[+1]
Legyenek a háromszög oldalai...
\end{feladat}

\section{Második section}
\begin{defin}[Második]
Egy háromszög...
\end{defin}
\begin{tet}[Pitagorasz]
Pitagorasz tétele...
\end{tet}
\begin{proof}
Bizonyítása...
\end{proof}
\begin{tet}[Pitagorasz tételének megfordítása]
Pitagorasz tételének megfordítása...
\end{tet}
\begin{megj}
Ez egy megjegyzés.
\end{megj}
\begin{lemma}
Ha egy háromszög...
\end{lemma}
\begin{feladat}[1]
Legyenek a háromszög oldalai...
\end{feladat}
\begin{feladat}[2]
Legyenek a háromszög oldalai...
\end{feladat}
\begin{feladat}[+1]
Legyenek a háromszög oldalai...
\end{feladat}



\end{document}