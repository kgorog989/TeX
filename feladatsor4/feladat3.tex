\documentclass{article}
\usepackage[magyar]{babel}
\usepackage{t1enc}
\usepackage{lipsum}
\usepackage{hulipsum}
\usepackage{amsthm}
\usepackage{listings}

\renewcommand{\lstlistingname}{Programkód}
\renewcommand{\lstlistlistingname}{Programkódok listája}

\begin{document}
\lstlistoflistings

\begin{lstlisting}[float=hb!,caption={Bináris keresés Javaban},
label=lst:buborek,language=java,tabsize=6,numbers=left,stepnumber=5,firstnumber=1, frame=leftline]
class BinarySearchExample{  
 public static void binarySearch(int arr[], int first, int last, int key){  
   int mid = (first + last)/2;  
   while( first <= last ){  
      if ( arr[mid] < key ){  
        first = mid + 1;     
      }else if ( arr[mid] == key ){  
        System.out.println("Element is found at index: " + mid);  
        break;  
      }else{  
         last = mid - 1;  
      }  
      mid = (first + last)/2;  
   }  
   if ( first > last ){  
      System.out.println("Element is not found!");  
   }  
 }  
 public static void main(String args[]){  
        int arr[] = {10,20,30,40,50};  
        int key = 30;  
        int last=arr.length-1;  
        binarySearch(arr,0,last,key);     
 }  
}  
\end{lstlisting}

\end{document}