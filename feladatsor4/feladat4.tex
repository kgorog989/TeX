\documentclass{article}
\usepackage[magyar]{babel}
\usepackage{t1enc}
\usepackage{lipsum}
\usepackage{hulipsum}
\usepackage{algpseudocode}
\usepackage{algorithm}


\begin{document}

\begin{algorithmic}[2]
\Procedure {Quicksort}{A,p,r}
\Require A írható tömb
\Require  1 $\leq$ p $\leq$ r $\leq$ Hossz[A] indexek
\Ensure a-b indextartományt rendezzük
\If {p < r}
\State       q $\leftarrow$ \Call{Partition}{A,p,r}
\State\Call{Quicksort}{A,p,q}
\State\Call{Quicksort}{A,q+1,r}
\EndIf
\EndProcedure
\Statex
\Statex
\Function {Partition}{A,p,r}
\Require A írható tömb
\Require  1 $\leq$ p $\leq$ r $\leq$ Hossz[A] indexek
\Ensure a-b indextartományt rendezzük
\State    x $\leftarrow$ A[p]
\State    i $\leftarrow$ p-1
\State    j $\leftarrow$ r+1
\While{true}
        \Repeat
\State            j $\leftarrow$ j-1
        \Until {A[j] <= x}
        \Repeat
\State            i $\leftarrow$ i+1
        \Until {A[i] >= x}
        \If {i == A[j]}
        \Else 
\State            return(j)
		\EndIf
\EndWhile
\EndFunction
\end{algorithmic}

\end{document}