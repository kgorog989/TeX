\documentclass{article}
\usepackage[magyar]{babel}
\usepackage{t1enc}
\usepackage{lipsum}
\usepackage{hulipsum}
\usepackage{algpseudocode}
\usepackage{algorithm}

\floatname{algorithm}{Algoritmus}
\renewcommand{\listalgorithmname}{Algoritmusok listája}
\algblockdefx{Do}{DoWhile}{\textbf{do}}[1]{\textbf{while}~#1~}

\algblockdefx{Switch}{EndSwitch}{\textbf{switch}}{\textbf{end switch}}
\algcblockdefx[Switch]{Switch}{Case}{EndSwitch}[1]{\textbf{~~~case~#1:}}{\textbf{end switch}}
\algcblockdefx{Switch}{Default}{EndSwitch}{\textbf{~~~default:}}{\textbf{end switch}}

\begin{document}
\listof{algorithm}{\listalgorithmname}

\begin{algorithm}
\begin{algorithmic}[2]
\Procedure {Quicksort}{@A,p,r}
\Require A írható tömb
\Require  1 $\leq$ p $\leq$ r $\leq$ Hossz[A] indexek
\Ensure a-b indextartományt rendezzük
\If {p < r}
\State       q $\leftarrow$ \Call{Partition}{@A,p,r}
\State\Call{Quicksort}{@A,p,q}
\State\Call{Quicksort}{@A,q+1,r}
\EndIf
\EndProcedure
\Statex
\Statex
\Function {Partition}{@A,p,r}
\Require A írható tömb
\Require  1 $\leq$ p $\leq$ r $\leq$ Hossz[A] indexek
\Ensure a-b indextartományt rendezzük
\State    x $\leftarrow$ A[p]
\State    i $\leftarrow$ p-1
\State    j $\leftarrow$ r+1
\While{true}
        \Repeat
\State            j $\leftarrow$ j-1
        \Until {A[j] $\leq$ x}
        \Repeat
\State            i $\leftarrow$ i+1
        \Until {A[i] $\geq$ x}
        \If {i == A[j]}
        \Else 
\State\Return{j}
		\EndIf
\EndWhile
\EndFunction
\end{algorithmic}
\caption{Gyorskeresés algoritmus}
\label{alg:quick}
\end{algorithm}


\begin{algorithm}
\begin{algorithmic}[2]
\Procedure {TesztASajatDefAlgnak}{x}
\Require x < 90
\Do
\State x += 1
\DoWhile{x<100}
\Statex
\Switch
\Case{1}
\State y = 2
\Case{2}
\State y = 7
\Default
\State x = 4
\State y = 10
\EndSwitch
\State x += y
\State\Return{x}

\EndProcedure
\end{algorithmic}
\caption{Teszt a saját blokkok definiálásához}
\label{alg:teszt}
\end{algorithm}

\end{document}