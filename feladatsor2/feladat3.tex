\documentclass{article}
\usepackage[magyar]{babel}
\usepackage{t1enc}
\usepackage{hulipsum}

\title{Szövegek}
\author{Görög Krisztina}
\date{\today}

\begin{document}

\renewcommand{\thefootnote}{\fnsymbol{footnote}}
\maketitle

\begin{abstract}
\hulipsum[1]\footnote{Ez egy megjegyzés.}
\end{abstract}

\setcounter{tocdepth}{5}
\pagenumbering{roman}
\tableofcontents
\clearpage

\setcounter{secnumdepth}{5}

\section{Első section}
\pagenumbering{arabic}
\subsection{Első subsection}
\hulipsum

\subsection{Második subsection}
\hulipsum

\section[Második]{Második section}
\subsection{Subsection}
\subsubsection{Subsubsection}
\paragraph{Paragraph}
\subparagraph{Subparagraph}

\appendix
\section{Függelék első section}
\subsection{Elsső subsection}
\quote{\hulipsum[2]}
\subsection{Második subsection}
\quotation{\hulipsum[2]}

\section{Függelék második section}
\subsection{Első subsection}
\begin{verse}
Hazádnak rendületlenűl\\
Légy híve, oh magyar;\\
Bölcsőd az s majdan sírod is,\\
Mely ápol s eltakar.\newline

A nagy világon e kivűl\\
Nincsen számodra hely;\\
Áldjon vagy verjen sors keze;\\
Itt élned, halnod kell.
\end{verse}
\subsection{Második subsection}


\end{document}