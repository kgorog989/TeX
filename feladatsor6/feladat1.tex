\documentclass{article}
\usepackage[magyar]{babel}
\usepackage{t1enc}
\usepackage{lipsum}
\usepackage{hulipsum}
\usepackage{amsmath}
\usepackage{amsfonts}
\usepackage{amssymb}
\usepackage{mathtools}
\usepackage{mathrsfs}
\usepackage{hyperref}


\begin{document}
\section{Bevezető}
Matematikai formulák: 

a) Az $\frac{1}{n^2} $ sor összege:

\[\sum_{i=1}^\infty\frac{1}{n^2}=\frac{\pi^2}{6}\]

b) Az $n!$ ($n$ faktoriális) a számok szorzata 1-től n-ig, azaz
\begin{equation}
n! := \prod_{i=1}^n k = 1 \cdot 2 \cdot \ldots \cdot n\text{.}
\label{eq:egyes}
\end{equation}

Konvekció szerint $0! = 1$.

c) Legyen $0 \leq k \leq n$. A binomiális együttható
\[\binom{n}{k} = \frac{n!}{k! \cdot (n-k)!}\text{,}\]

ahol a faktoriálist \eqref{eq:egyes} szerint definiáljuk.

d) Az előjel- azaz szignum függvényt a következőképpen definiáljuk:
\[\operatorname{sgn}(x) := \begin{cases}
1, & \text{ha } x > 0, \\
0, & \text{ha } x = 0, \\
-1, & \text{ha } x < 0.
\end{cases}\]

\section{Determináns}
Matematikai formulák: 

a) Legyen
\[[n] := \lbrace1, 2, \ldots, n\rbrace\]

a természetes számok halmaza 1-től n-ig.

b) Egy $n$-edrendű $\text{permutáció }\sigma$ egy bijekció $[n]$-ből $[n]$-be. Az $n$-edrendű permutációk halmazát, az ún. szimmetrikus csoportot, $S_n$-nel jelöljük.

c) Egy $\sigma \in S_n$ permutációban inverziónak nevezünk egy $(i, j)$ párt, ha $i < j$
de $\sigma i > \sigma j$.

d) Egy $\sigma \in S_n$ permutáció paritásának az inverziók számát nevezzük:
\[ \mathcal{I}(\sigma) := \vert\lbrace(i, j) \vert i, j \in [n], i < j, \sigma_i > \sigma_j \rbrace\vert.\]

e) Legyen $A \in \mathbb{R}^{n \times n} \text{, egy }n \times n$-es (négyzetes) valós mátrix:
\[ A = \left( \begin{matrix}
a_11 & a_12 & \cdots & a_1n \\
a_21 & a_22 & \cdots & a_2n \\
\vdots & \vdots & \ddots & \vdots \\
a_n1 & a_n2 & \cdots & a_nn \\
\end{matrix} \right) \]

Az $A$ mátrix determinánsát a következőképpen definiáljuk:
\begin{equation}
\operatorname{det}(A) =  \begin{vmatrix}
a_11 & a_12 & \cdots & a_1n \\
a_21 & a_22 & \cdots & a_2n \\
\vdots & \vdots & \ddots & \vdots \\
a_n1 & a_n2 & \cdots & a_nn \\
\end{vmatrix} := \sum_{\sigma \in S_n} (-1)^{\mathcal{I}(\sigma)} \prod_{i=1}^n a_{i\sigma_i}
\label{kettes}
\end{equation}

\section{Logikai azonosság}

Tekintsük az $L = \lbrace0, 1\rbrace$ halmazt, és rajta a következő, igazságtáblával definiált műveleteket:

\[\begin{array}{c || c}
x & \bar{x} \\ \hline
0 & 1 \\
1 & 0 \\
\end{array}
\quad
\begin{array}{c c || c | c | c}
x & y & x \vee y & x \wedge y & x \rightarrow y \\ \hline
0 & 0 & 0 & 0 & 1 \\
0 & 1 & 1 & 0 & 1 \\
1 & 0 & 1 & 0 & 0 \\
1 & 1 & 1 & 1 & 1 \\
\end{array}
\]

Legyenek $a, b, c, d \in L$. Belátjuk a következő azonosságot:

\begin{equation}
(a \wedge b \wedge c) \rightarrow d = a \rightarrow ( b \rightarrow( c \rightarrow d))\text{.}
\label{eq:harom}
\end{equation}

A következő azonosságokat bizonyítás nélkül használjuk:

\begin{subequations}
\begin{equation}
x \rightarrow y = \bar{x} \vee y
\label{eq:negyes_a}
\end{equation}
\begin{equation}
\overline{x \vee y} = \bar{x} \wedge \bar{y}
\quad
\overline{x \wedge y} = \bar{x} \vee \bar{y}
\label{eq:negyes_b}
\end{equation}
\label{eq:negyes}
\end{subequations}

A \eqref{eq:harom} bal oldala, \eqref{eq:negyes} felhasználásával

\begin{equation}
(a \wedge b \wedge c) \rightarrow d \underset{\eqref{eq:negyes_a}}{=} \overline{a \wedge b \wedge c} \vee d \underset{\eqref{eq:negyes_b}}{=} (\bar{a} \vee \bar{b} \vee \bar{c}) \vee d \text{.}
\label{eq:otos}
\end{equation}

A \eqref{eq:harom} jobb oldala, \eqref{eq:negyes_a} ismételt felhasználásával
\begin{align}
a \rightarrow ( b \rightarrow( c \rightarrow d)) &= \bar{a} \vee (b \rightarrow(c \rightarrow d)) \\ \nonumber
&= \bar{a} \vee ( \bar{b} \vee (c \rightarrow d)) \\ \nonumber
&= \bar{a} \vee ( \bar{b} \vee (\bar{c} \vee d)), \\ \nonumber \\ \nonumber
\label{eq:hatos}
\end{align}

ami a $\vee$ asszociativitása miatt egyenlő \eqref{eq:otos} egyenlettel.

\section{Binomiális tétel}

\begin{subequations}
\begin{align}
(a+b)^{n+1} &= (a+b) \cdot \left( \sum_{k=0}^n \binom{n}{k} a^{n-k}b^k \right) \\
&= \cdots \\ \nonumber
&= \sum_{k=0}^n \binom{n}{k} a^{(n+1)-k}b^k 
+ \sum_{k=1}^{n+1} \binom{n}{k-1} a^{(n+1)-k}b^{k} \\
&= \cdots \\ \nonumber
\begin{split}
&= \binom{n+1}{0} a^{n+1-0} b^0 + \sum_{k=1}^n \binom{n+1}{k} a^{(n+1)-k}b^k \\ &+ \binom{n+1}{n+1} a^{n+1-(n+1)} b^{n+1} 
\end{split} \\
&= \sum_{k=0}^{n+1} \binom{n+1}{k} a^{(n+1)-k}b^k 
\end{align}


\end{subequations}

\end{document}